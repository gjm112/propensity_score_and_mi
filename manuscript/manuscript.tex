\documentclass{article}

\usepackage{arxiv}

\usepackage[utf8]{inputenc} % allow utf-8 input
\usepackage[T1]{fontenc}    % use 8-bit T1 fonts
\usepackage{lmodern}        % https://github.com/rstudio/rticles/issues/343
\usepackage{hyperref}       % hyperlinks
\usepackage{url}            % simple URL typesetting
\usepackage{booktabs}       % professional-quality tables
\usepackage{amsfonts}       % blackboard math symbols
\usepackage{nicefrac}       % compact symbols for 1/2, etc.
\usepackage{microtype}      % microtypography
\usepackage{graphicx}

\title{\textbf{Estimating Aging Curves: Using Multiple Imputation to Examine Career Trajectories of MLB Offensive Players}}

\author{
    Quang Nguyen
   \\
    Department of Statistics and Data Science \\
    Carnegie Mellon University \\
  Pittsburgh, PA 15213 \\
  \texttt{\href{mailto:nmquang@cmu.edu}{\nolinkurl{nmquang@cmu.edu}}} \\
   \And
    Gregory J. Matthews
   \\
    Department of Mathematics and Statistics \\
    Loyola University Chicago \\
  Chicago, IL 60660 \\
  \texttt{\href{mailto:gmatthews1@luc.edu}{\nolinkurl{gmatthews1@luc.edu}}} \\
  }


% tightlist command for lists without linebreak
\providecommand{\tightlist}{%
  \setlength{\itemsep}{0pt}\setlength{\parskip}{0pt}}


% Pandoc citation processing
\newlength{\cslhangindent}
\setlength{\cslhangindent}{1.5em}
\newlength{\csllabelwidth}
\setlength{\csllabelwidth}{3em}
\newlength{\cslentryspacingunit} % times entry-spacing
\setlength{\cslentryspacingunit}{\parskip}
% for Pandoc 2.8 to 2.10.1
\newenvironment{cslreferences}%
  {}%
  {\par}
% For Pandoc 2.11+
\newenvironment{CSLReferences}[2] % #1 hanging-ident, #2 entry spacing
 {% don't indent paragraphs
  \setlength{\parindent}{0pt}
  % turn on hanging indent if param 1 is 1
  \ifodd #1
  \let\oldpar\par
  \def\par{\hangindent=\cslhangindent\oldpar}
  \fi
  % set entry spacing
  \setlength{\parskip}{#2\cslentryspacingunit}
 }%
 {}
\usepackage{calc}
\newcommand{\CSLBlock}[1]{#1\hfill\break}
\newcommand{\CSLLeftMargin}[1]{\parbox[t]{\csllabelwidth}{#1}}
\newcommand{\CSLRightInline}[1]{\parbox[t]{\linewidth - \csllabelwidth}{#1}\break}
\newcommand{\CSLIndent}[1]{\hspace{\cslhangindent}#1}

\usepackage{setspace} \setstretch{1.07} \usepackage{float} \floatplacement{figure}{H} \usepackage{mathpazo}
\begin{document}
\maketitle


\begin{abstract}
In sports, an aging curve depicts the relationship between average
performance and age in athletes' careers. This paper investigates the
aging curves for offensive players in the Major League Baseball. We
study this problem in a missing data context and account for different
types of dropouts of baseball players during their careers. In
particular, the performance metrics associated with the missing seasons
are imputed using a multiple imputation model for multilevel data, and
the aging curves are constructed based on the imputed datasets. We first
perform a simulation study to evaluate the effects of different dropout
mechanisms on the estimation of aging curves. Our method is then
illustrated with analyses of MLB player data from past seasons. Results
suggest an overestimation of the aging curves constructed without
imputing the unobserved seasons, whereas a better estimate is achieved
with our approach.
\end{abstract}

\keywords{
    aging curve
   \and
    baseball
   \and
    multiple imputation
   \and
    statistics in sports
  }

\hypertarget{sec:intro}{%
\section{Introduction}\label{sec:intro}}

Mitra \& Reiter (2016): This paper blah blah

\hypertarget{supplementary-material}{%
\section*{Supplementary Material}\label{supplementary-material}}
\addcontentsline{toc}{section}{Supplementary Material}

All code for reproducing the analyses in this paper is publicly
available at

\hypertarget{references}{%
\section*{References}\label{references}}
\addcontentsline{toc}{section}{References}

\hypertarget{refs}{}
\begin{CSLReferences}{1}{0}
\leavevmode\vadjust pre{\hypertarget{ref-MitraReiter2016}{}}%
Mitra, R., \& Reiter, J. (2016). A comparison of two methods of
estimating propensity scores after multiple imputation.
\emph{Statistical Methods in Medical Research}, \emph{25}(1), 188--204.
\url{https://doi.org/10.1177/0962280212445945}

\end{CSLReferences}

\bibliographystyle{unsrt}
\bibliography{references.bib}


\end{document}
